\documentclass[11pt]{article}

% ---------- Packages ----------
\usepackage[a4paper, margin=1in]{geometry}
\usepackage{fancyhdr}
\usepackage{titlesec}
\usepackage{minted}
\usepackage{graphicx}
\usepackage{amsmath, amssymb}
\usepackage{hyperref}
\usepackage{xcolor}
\usepackage{enumitem}
\usepackage{tikz}
\usepackage{caption}
\usepackage{listings}
\usepackage{pagecolor}
\usepackage{colortbl} % Required for \rowcolors
\usepackage{inconsolata} % Monospaced font
\usepackage[most]{tcolorbox}
\usepackage{amsmath}
\usepackage{booktabs}
\tcbuselibrary{skins, breakable, theorems}


\setlength{\parskip}{0.5em}
\setlength{\parindent}{0pt}

% ---------- Dark Theme ----------
\definecolor{backgroundblack}{HTML}{1E1E1E}
\definecolor{textwhite}{HTML}{D4D4D4}
\definecolor{linkblue}{HTML}{569CD6}
\definecolor{codegray}{gray}{0.12}

\pagecolor{backgroundblack}
\color{textwhite}

\hypersetup{
    colorlinks=true,
    linkcolor=linkblue,
    urlcolor=linkblue
}
%----------- Custom Reminder Box ----------


\newtcolorbox{reminderbox}{
    enhanced,
    breakable,
    colback=yellow!10,
    colframe=orange!80!red,
    coltitle=black,
    title= Reminder,
    fonttitle=\bfseries,
    boxrule=0.8pt,
    % Removed unsupported skin and size options
    sharp corners=south,
    left=5pt,
    right=5pt,
    top=5pt,
    bottom=5pt
  }


% ---------- Header/Footer ----------
\pagestyle{fancy}
\fancyhf{}
\fancyhead[L]{\textcolor{textwhite}{Programming Systems 1}}
\fancyhead[R]{\textcolor{textwhite}{\leftmark}}
\fancyfoot[C]{\textcolor{textwhite}{\thepage}}

% ---------- Title ----------
\title{\vspace{1in} \Huge \textbf{\textcolor{linkblue}{Programming Systems 1}} \\[0.5em] \LARGE Lecture Notes \vspace{0.5in}}
\author{\Large \textcolor{textwhite}{Samuel Montalbano} \\ \large Department of Computer Science \\ \large University of Denver}
\date{\textcolor{textwhite}{\today}}

% ---------- Section Styling ----------
\titleformat{\section}
  {\color{linkblue}\normalfont\Large\bfseries}{\thesection}{1em}{}
\titleformat{\subsection}
  {\color{linkblue}\normalfont\large\bfseries}{\thesubsection}{1em}{}
\titleformat{\subsubsection}
  {\color{linkblue}\normalfont\normalsize\bfseries}{\thesubsubsection}{1em}{}

% ---------- Code Styling ----------
\setminted{
    fontsize=\small,
    bgcolor=codegray,
    linenos,
    breaklines,
    frame=lines,
    framesep=2mm,
    style=monokai
}

% ---------- Begin Document ----------
\begin{document}

\maketitle
\newpage
\tableofcontents
\newpage


\section{C: Cheat Sheet}

\subsection{Basic Syntax}
\begin{itemize}
    \item \texttt{\#include <stdio.h>} - Include standard I/O library.
    \item \texttt{int main()} - Entry point of the program.
    \item \texttt{return 0;} - Indicates successful program termination.
    \item \texttt{// Comment} - Single-line comment.
    \item \texttt{/* Comment */} - Multi-line comment.
\end{itemize}

\subsection{Data Types}
\begin{itemize}
    \item \texttt{int} - Integer (e.g., \texttt{int x = 5;}).
    \item \texttt{float} - Floating-point number (e.g., \texttt{float y = 3.14;}).
    \item \texttt{double} - Double precision floating-point number.
    \item \texttt{char} - Single character (e.g., \texttt{char c = 'A';}).
    \item \texttt{bool} - Boolean (requires \texttt{\#include <stdbool.h>}).
\end{itemize}

\subsection{Control Structures}
\begin{itemize}
    \item \texttt{if (condition) \{...\}} - Conditional statement.
    \item \texttt{else if (condition) \{...\}} - Additional condition.
    \item \texttt{else \{...\}} - Default block.
    \item \texttt{for (init; condition; increment) \{...\}} - For loop.
    \item \texttt{while (condition) \{...\}} - While loop.
    \item \texttt{do \{...\} while (condition);} - Do-while loop.
    \item \texttt{switch (variable) \{ case value: ...; break; \}} - Switch statement.
\end{itemize}

\subsection{Functions}
\begin{itemize}
    \item Declaration: \texttt{int add(int a, int b);}
    \item Definition: \texttt{int add(int a, int b) \{ return a + b; \}}
    \item Call: \texttt{int result = add(5, 3);}
\end{itemize}

\subsection{Pointers}
\begin{itemize}
    \item Declaration: \texttt{int *p;}
    \item Assign address: \texttt{p = \&x;}
    \item Dereference: \texttt{*p = 10;}
\end{itemize}

\subsection{Memory Management}
\begin{itemize}
    \item \texttt{malloc(size)} - Allocate memory.
    \item \texttt{calloc(num, size)} - Allocate and initialize memory.
    \item \texttt{realloc(ptr, size)} - Resize memory block.
    \item \texttt{free(ptr)} - Deallocate memory.
\end{itemize}

\subsection{Common Format Specifiers}
\begin{itemize}
    \item \texttt{\%d} - Integer.
    \item \texttt{\%f} - Float.
    \item \texttt{\%c} - Character.
    \item \texttt{\%s} - String.
    \item \texttt{\%p} - Pointer.
\end{itemize}

\subsection{Common Escape Sequences}
\begin{itemize}
    \item \texttt{\textbackslash n} - Newline.
    \item \texttt{\textbackslash t} - Tab.
    \item \texttt{\textbackslash \textbackslash} - Backslash.
    \item \texttt{\textbackslash "} - Double quote.
\end{itemize}

\subsection{Best Practices}
\begin{itemize}
    \item Always initialize variables.
    \item Use meaningful names for variables and functions.
    \item Check return values of functions like \texttt{malloc} and \texttt{scanf}.
    \item Avoid magic numbers; use constants or \texttt{\#define}.
    \item Comment your code for clarity.
\end{itemize}






\section{Class 2: Introduction to C}

\subsection{Basics of C Programming}
\begin{minted}{c}
// Basics of C programming
#include <stdio.h>

int main() {
    int x; // Declare type of variables
    x = 5;
    return 0;
}
\end{minted}

\subsection{Primitive Data Types}
Primitive data types in C:

\begin{table}[h!]
\centering
\rowcolors{2}{codegray}{}
\renewcommand{\arraystretch}{1.3}
\begin{tabular}{|c|c|c|}
\hline
\textbf{Data Type} & \textbf{Description} & \textbf{Example} \\ \hline
\texttt{char}      & Character type       & \texttt{char c = 'A';} \\ \hline
\texttt{short}     & Short integer type   & \texttt{short s = 10;} \\ \hline
\texttt{int}       & Integer type         & \texttt{int x = 5;} \\ \hline
\texttt{long}      & Long integer type    & \texttt{long l = 100000L;} \\ \hline
\texttt{float}     & Floating-point type  & \texttt{float y = 3.14f;} \\ \hline
\texttt{double}    & Double precision type & \texttt{double d = 3.14159;} \\ \hline
\texttt{Bool}      & Boolean type         & \texttt{Bool b = 1;} \\ \hline
\end{tabular}
\caption{Primitive Data Types in C}
\label{tab:datatypes}
\end{table}

\begin{itemize}
    \item Chars and ints are interoperable
    \item No byte type in C
    \item Huge negative numbers may mean you never initialized your variable
    \item To print two stored integers: \texttt{printf("\%d \%d", x, y);}
    \item Float: \texttt{printf("\%f", x);}
    \item Double: \texttt{printf("\%lf", x);}
    \item Char: \texttt{printf("\%c", x);}
    \item String: \texttt{printf("\%s", x);}
    \item Decimal: \texttt{printf("\%d", x);}
    \item Pointer: \texttt{printf("\%p", x);}
    \item Hexadecimal: \texttt{printf("\%x", x);}
\end{itemize}

\subsection{If Statements}
\begin{minted}{c}
#include <stdio.h>

int main() {
    int x = 5;
    if (x > 0) {
        printf("x is positive\n");
    } else if (x < 0) {
        printf("x is negative\n");
    } else {
        printf("x is zero\n");
    }
    return 0;
}
\end{minted}

\begin{itemize}
    \item \texttt{if} evaluates a condition and runs its block if true
    \item \texttt{else if} adds another condition
    \item \texttt{else} runs if all other conditions are false
    \item \texttt{switch} allows multiple fixed values
    \item \texttt{case} defines the individual branches
\end{itemize}

\subsection{Loops}

\subsubsection{For Loop}
\begin{minted}{c}
#include <stdio.h>
int main() {
    for (int i = 0; i < 5; i++) {
        printf("Iteration %d\n", i);
    }
    return 0;
}
\end{minted}

\subsubsection{While Loop}
\begin{minted}{c}
#include <stdio.h>
int main() {
    int i = 0;
    while (i < 5) {
        printf("Iteration %d\n", i);
        i++;
    }
    return 0;
}
\end{minted}

\subsubsection{Do While Loop}
\begin{minted}{c}
#include <stdio.h>
int main() {
    int i = 0;
    do {
        printf("Iteration %d\n", i);
        i++;
    } while (i < 5);
    return 0;
}
\end{minted}

\subsection{Functions}
\begin{minted}{c}
#include <stdio.h>
int add(int a, int b) {
    return a + b;
}
int main() {
    int result = add(5, 3);
    printf("Result: %d\n", result);
    return 0;
}
\end{minted}

\begin{itemize}
    \item Warning: Implicit declaration of function — define before use
\end{itemize}

\subsubsection{Function Prototypes}
\begin{minted}{c}
#include <stdio.h>
int add(int a , int b); // Function prototype
int main() {
    int result = add(5, 3);
    printf("Result: %d\n", result);
    return 0;
}
int add(int a, int b) { // Function definition
    return a + b;
}
\end{minted}
\newpage

\subsection{Arrays}
A process is a running instance of a program. It has its own memory and resources.
An array is a collection of elements of the same type, stored in contiguous memory locations.
\begin{minted}{c}
#include <stdio.h>
int main() {
    int arr[5] = {1, 2, 3, 4, 5}; // Declare and initialize an array
    for (int i = 0; i < 5; i++) {
        printf("Element %d: %d\n", i, arr[i]);
    }
    return 0;
}
\end{minted}
\begin{itemize}
    \item Arrays are zero-indexed
    \item Use \texttt{sizeof(arr)/sizeof(arr[0])} to get the number of elements
    \item Multidimensional arrays: \texttt{int arr[2][3];}
    \item Accessing elements: \texttt{arr[i][j];}
    \item Arrays are passed by reference
    \item To pass an array to a function, use the syntax \texttt{void func(int arr[]);}
    \item To pass a multidimensional array, use \texttt{void func(int arr[][3]);}
    \item To pass a pointer to an array, use \texttt{void func(int *arr);}
    \item To pass a pointer to a multidimensional array, use \texttt{void func(int (*arr)[3]);}
    \item To pass a pointer to a pointer, use \texttt{void func(int **arr);}
\end{itemize}

\subsection{Pointers: AHHHHHHHHHHHHHHHHHHHHHHHHHHHH}
\begin{reminderbox}
    Learn Pointers, Learn Pointers, Learn Pointers, Learn Pointers!!!!!!!!!!!!!!!!!!!
    \end{reminderbox}
Pointers are variables that store the address of another variable. They are used for dynamic memory allocation, arrays, and functions.
\begin{minted}{c}
#include <stdio.h>
int main() {
    int x = 5;
    int *p = &x; // Declare a pointer and assign it the address of x
    printf("Value of x: %d\n", *p); // Dereference the pointer to get the value of x
    return 0;
}
\end{minted}
\begin{itemize}
    \item Pointers are declared with an asterisk (\texttt{*})
    \item The address operator (\texttt{\&}) is used to get the address of a variable
    \item The dereference operator (\texttt{*}) is used to access the value at the address stored in the pointer
    \item Pointers can be used to create dynamic arrays
    \item Use \texttt{malloc} to allocate memory for a pointer
    \item Use \texttt{free} to deallocate memory for a pointer
\end{itemize}

\newpage






\section{Class 3: Pointer infomration}

\subsection{Pointers written out}
\begin{minted}{c}
// Basics of C programming
#include <stdio.h>

int main() {
    int x; // Declare type of variables
    x = 5;
    int *px=&x; // Declare a pointer to x
    int y=*px; // Declare y and assign it the value of x
    return 0;
}
\end{minted}

\begin{minted}{c}
    // Basics of C programming
    #include <stdio.h>
    
    int main() {
        int x; // Declare type of variables
        x = 5;
        int *px=&x; // Declare a pointer to x
        return 0;
    }
    \end{minted}





\section{Class 4: header/source files, incremental compilation, memory allocation, and garbage collection}
\subsection{Incremental}
\begin{minted}{c}
// Basics of C programming
#include <stdio.h>
int doubleIt(int x){
    return x*2;
}
int main() {
    int a=5
    int b=doubleIt(A)
    printf("%d\n", b);

    return 0;
}

\end{minted}

Note that we can store functions like this in another file and reference it.
To do so:
\begin{itemize}
    \item You can do gcc file 1 and file 2 -o run name. This executes the files directly.
    \item Or you can use header files. Putting prototypes in a .h file. For example, double.h. Use \#include to include the .h file in the file with your main.
    \item Either put .h in a standard location (Don't do that).
    \item Put "" around the .h to show it's in the directory that you are currently in.
    \item Good practice is putting structures above the prototypes.
    \item 
\end{itemize}
To recompile a singular file:
gcc -c file change.c
gcc week3.o double.o -o foo
./filename

% In .h file include \\#ifndef DOUBLE_H
% \\#define DOUBLE_H 
% //prototype
% \\#endif

To automate the compile process:
use makefile

\begin{minted}{makefile}
    # makefile
    foo:
        gcc week3.o double.o -o foo

    week3.o:
        gcc -c week3.c

    double.o:
        gcc -c double.c

    clean:
        rm *.o 
        rm foo
\end{minted}

Using makefile:
    use make foo to update file.

\subsection{Information on Memory Allocator Lab}

\section{Class 5: C++}
Rational.h
\begin{minted}{c}
    int main(){
        class Rational{
            private:
            int num;
            int dem;
            public:

        };
    
    char *
    }
    
\end{minted}

Rational.cc
\begin{minted}{c++}
    int main(){
    
    char *
    }
    
\end{minted}

ratMain.cc
\begin{minted}{c++}
    int main(){
    
    char *
    }
    
\end{minted}

\section{Class 6: Bit Manipulation}

\subsection{Bitwise}
\begin{minted}{c}
#include <stdio.h>
int main(){
    unsigned int a=5;
    unsigned int b=6;
    unsigned int c=a&1; // binary truth table. bitwise and.
    unsigned int x=a|b ;//or
    unsigned int x=a^b; //exclusive or. 1 or the other
    unsigned int x=~a; //not 
    unsigned int aa=6;
    unsigned int bb=aa <<1; //bit shifting. Left Shift 1 bit. Also >>
    
    printf("%d\n",c);
    char x=127;
    char y=x+1;
    print("%d\n,y");
return 0;    
}
    
\end{minted}


\subsection{2s-compliment}




\begin{center}
\begin{tabular}{cccc}
\toprule
Binary & Unsigned Value & Signed (Sign-Magnitude) & Two's Complement \\
\midrule
000 & 0 & 0 & 0 \\
001 & 1 & 1 & 1 \\
010 & 2 & 2 & 2 \\
011 & 3 & 3 & 3 \\
100 & 4 & -0 & -4 \\
101 & 5 & -1 & -3 \\
110 & 6 & -2 & -2 \\
111 & 7 & -3 & -1 \\
\bottomrule
\end{tabular}
\end{center}





    
\section*{Appendix}
\addcontentsline{toc}{section}{Appendix}

\subsection*{Format Specifiers in \texttt{printf}}
\begin{itemize}
    \item \texttt{\%d} - Decimal integer
    \item \texttt{\%f} - Floating-point number
    \item \texttt{\%lf} - Double precision floating-point number
    \item \texttt{\%c} - Single character
    \item \texttt{\%s} - String
    \item \texttt{\%p} - Pointer address
    \item \texttt{\%x} - Hexadecimal integer
    \item \texttt{\%o} - Octal integer
\end{itemize}

\subsection*{Common Escape Sequences in C}
\begin{itemize}
    \item \texttt{\textbackslash n} - Newline
    \item \texttt{\textbackslash t} - Horizontal tab
    \item \texttt{\textbackslash r} - Carriage return
    \item \texttt{\textbackslash \textbackslash} - Backslash
    \item \texttt{\textbackslash "} - Double quote
    \item \texttt{\textbackslash 0} - Null character
\end{itemize}

\subsection*{Memory Allocation Functions}
\begin{itemize}
    \item \texttt{malloc(size\_t size)} - Allocates memory of specified size and returns a pointer to it.
    \item \texttt{calloc(size\_t num, size\_t size)} - Allocates memory for an array of \texttt{num} elements, each of \texttt{size} bytes, and initializes all bytes to zero.
    \item \texttt{realloc(void *ptr, size\_t size)} - Resizes the memory block pointed to by \texttt{ptr} to \texttt{size} bytes.
    \item \texttt{free(void *ptr)} - Frees the memory block pointed to by \texttt{ptr}.
\end{itemize}

\subsection*{Pointer Arithmetic}
\begin{itemize}
    \item Adding an integer to a pointer moves the pointer forward by that many elements.
    \item Subtracting an integer from a pointer moves the pointer backward by that many elements.
    \item Subtracting two pointers gives the number of elements between them.
\end{itemize}

\subsection*{Common Compiler Warnings and Errors}
\begin{itemize}
    \item \texttt{implicit declaration of function} - Ensure the function is declared before use or include the appropriate header file.
    \item \texttt{segmentation fault} - Check for invalid memory access, such as dereferencing a null or uninitialized pointer.
    \item \texttt{array index out of bounds} - Ensure array indices are within valid range.
    \item \texttt{unused variable} - Remove or use declared variables to avoid warnings.
\end{itemize}

\subsection*{Multidimensional Arrays}
\begin{itemize}
    \item Declaration: \texttt{int arr[rows][cols];}
    \item Accessing elements: \texttt{arr[i][j];}
    \item Passing to functions: \texttt{void func(int arr[][cols]);}
\end{itemize}

\subsection*{Best Practices for Writing C Code}
\begin{itemize}
    \item Always initialize variables before use.
    \item Use meaningful variable and function names.
    \item Comment your code to explain complex logic.
    \item Avoid magic numbers; use \texttt{\#define} or constants.
    \item Check the return value of functions like \texttt{malloc} and \texttt{scanf}.
    \item Use \texttt{const} for variables that should not be modified.
\end{itemize}



\end{document}
